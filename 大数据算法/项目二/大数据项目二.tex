\documentclass[UTF8]{ctexart}
\usepackage{CJKutf8}
\usepackage{graphicx}
\usepackage{amsmath}
\usepackage{diagbox}
\usepackage{hyperref}
\usepackage{xcolor}
\usepackage[linesnumbered,boxed,ruled,commentsnumbered]{algorithm2e}
\usepackage{cite}

  \author{黄晃\ 数院 1701210098 }
  \title{大数据项目二}
\begin{document}

  \maketitle
  \section{问题}
  原问题为
  \begin{equation}\label{p:1}
    \begin{split}
       min\  &  \frac{1}{n}f_i(w)+\lambda\|w\|_1 \\
    \end{split}
  \end{equation}
  where $f_i(w)=log(1+exp(-y^iw^Tx^i)and lambda>0$
  \paragraph{}
  为了使用stochastic optimization,可以将问题改写成
    \begin{equation}\label{p:2}
    \begin{split}
       min\  &  \frac{1}{n}f_i(w)\\
    \end{split}
  \end{equation}
  where $f_i(w)=log(1+exp(-y^iw^Tx^i)+\lambda\|w\|_1$
  \subsection{数据集}
    使用要求的两个数据集:MINIST和Coverttype.
    \paragraph{MINIST}
    一共70000个样本,其中每个$x^i$是28*28的灰度矩阵向量化的结果,而$y^i$是该幅图片对应的label
    \paragraph{Covertype}
    一共581012个样本,其中每个$x^i$是54维向量
    \paragraph{数据归一化}
    对于$X=[x_1,x_2,\cdots,x_N]=[a_1,a_2,\cdots,a_p]^T$,我们对其进行行归一化,即取$\hat{X}=[\hat{a_1},\hat{a_2},\cdots,\hat{a_p}]^T$,其中$\hat{a_i}=a_i/max(a_i)$
    \subsection{算法}
  一共实现了要求的两个算法:SAG和SVRG以及附加任务:使用BB步长的SG方法


  \subsubsection{SGD}
  在~\cite{bottou2010large}中介绍了随机梯度方法,即将正常的梯度下降
  $$
  w_{k+1} = w_k-\alpha_k \nabla f(x_k)
  $$
  改成了随机版本
  $$
  w_{k+1} = w_k-\alpha_k \nabla f_{s_k}(x_k)
  $$
  其中$s_k$随机取样于$\{1,2,\cdots,n\}$
  \subsubsection{SAG}
  在SGD的基础上,文献~\cite{schmidt2017minimizing}提出了一个平均的思想,即将下降修改成了
  $$w_{k+1}=w_k-\alpha_k(\frac{1}{n}(\nabla f_{s_k}(w_k)-g_{k+1}^{s_k})+\frac{1}{n}\sum_{i=1}^{N}g_{k-1}^i)$$
  其中
  \begin{equation*}
    g_k^i= \left\{
    \begin{array}{cc}
      \nabla f_{i}(w_k) &if\ i=s_k\\
      g_{k+1}^{s_k} & else
    \end{array}
    \right.
  \end{equation*}
  所以需要存储每个$f_i$随着随机量$s_K$最近一次被计算时得到的值作为$g_k^i$.此外,为了启动算法,在第一次迭代开始前,对每个$f_i$计算一次相应的次梯度



  \subsubsection{SVRG}

\begin{algorithm}
    \SetAlgoNoLine % 不要算法中的竖线
    \SetKwInOut{Input}{\textbf{输入}}\SetKwInOut{Output}{\textbf{输出}} % 替换关键词

    \Input{
        \\
        步长 $\eta$\;\\
        迭代数 $T$\;\\
        内循环次数 $m$ \;\\
        初值 $\hat{x_0}$\;\\}
    \Output{
        \\
        $\hat{x_T}$\;\\}
    \BlankLine

    \For {$j=1,2,\cdots,T$}{
        $g_j=\nabla f(\hat{x_{j-1}})=\frac{1}{n}\sum_{i=1}^{n}\nabla f_i(\hat{x_{j-1}})$ \;
        $x_0^{(j)}\leftarrow \hat{x_{j-1}}$ \;
        $N_j\leftarrow m$ \;
        \For {$k=1,2,\cdots,N_j$}{
            Random pick $i_k\in [n]$\;
            $\nu_{k-1}^{(j)}\leftarrow \nabla f_{i_k}(x_{k-1}^{(j)})-f_{i_k}(x_{0}^{(j)})+g_j$\;
            $ x_k^{(j)\leftarrow x_{k-1}^{(j)}}-\eta \nu_{k-1}^{(j)} $ \;
        }
        $\hat{x_j}\leftarrow x_{N_j}^{(j)}$\;
    }
    \caption{Stochastic Variance Reduced Gradient(SVRG)\label{al1}}
\end{algorithm}
  每求一次完整的函数f的梯度,进行m次随机梯度的迭代,然后将结果的平均作为外部循环的结果.每次内循环的下降方向选择为
  $$
  v_k=\nabla f_{s_k}(w_k)-\nabla f_{s_k}(y)+\nabla f(w_i)
$$
其中$w_i$为外循环第i步的值
\paragraph{参数}
内循环次数m=2n.
\subsubsection{MSVRG}
在~\cite{reddi2016stochastic}中,考虑到并行性以及通信的成本,对SVRG的内循环部分进行了改进,在第7行增加了一个mini batch的集合$I_k$,然后求这个集合的f的梯度而不是单独某一个
\begin{algorithm}
    \SetAlgoNoLine % 不要算法中的竖线
    \SetKwInOut{Input}{\textbf{输入}}\SetKwInOut{Output}{\textbf{输出}} % 替换关键词

    \Input{
        \\
        步长 $\eta$\;\\
        迭代数 $T$\;\\
        内循环次数 $m$ \;\\
        初值 $\hat{x_0}$\;\\}
    \Output{
        \\
        $\hat{x_T}$\;\\}
    \BlankLine

    \For {$j=1,2,\cdots,T$}{
        $g_j=\nabla f(\hat{x_{j-1}})=\frac{1}{n}\sum_{i=1}^{n}\nabla f_i(\hat{x_{j-1}})$ \;
        $x_0^{(j)}\leftarrow \hat{x_{j-1}}$ \;
        $N_j\leftarrow m$ \;
        \For {$k=1,2,\cdots,N_j$}{
            mini batch $I_k\in [n]$\;
            $\nu_{k-1}^{(j)}\leftarrow \sum_{i_k\in I_k}(\nabla f_{i_k}(x_{k-1}^{(j)})-f_{i_k}(x_{0}^{(j)}))+g_j$\;
            $ x_k^{(j)\leftarrow x_{k-1}^{(j)}}-\eta \nu_{k-1}^{(j)} $ \;
        }
        $\hat{x_j}\leftarrow x_{N_j}^{(j)}$\;
    }
    \caption{Minibatch Stochastic Variance Reduced Gradient(MSVRG)\label{al2}}
\end{algorithm}
\paragraph{参数选择}
根据文献理论结果,minibatch大小为10,内部循环次数m取成$\frac{n}{10}$
  \subsubsection{SAGA}
  在文献~\cite{defazio2014saga}中,结合了SVRG的优点,将SAG方法的更新方向修改成无偏的版本
  $$w_{k+1}=w_k-\alpha_k(\nabla f_{s_k}(w_k)-g_{k+1}^{s_k}+\frac{1}{n}\sum_{i=1}^{N}g_{k-1}^i)$$
\subsubsection{SCSG}
为了支持非强凸问题,在献~\cite{lei2016less}对SVRG的第二行进行了改进,在计算整体梯度$g_j$时,不再对所有的$f_i$求梯度,而是去了其中大小为B的一部分.然后内循环次数$N_j\sim Geom(\frac{B}{B+1})$.但同时作者表明实验中,取固定的$N_j=B$的效果一样;内循环$i\in [n]$与$i \in I_j$表现也相当.因此,为了节省计算量,我们在这两点上都选择了后一种方式.
\begin{algorithm}
    \SetAlgoNoLine % 不要算法中的竖线
    \SetKwInOut{Input}{\textbf{输入}}\SetKwInOut{Output}{\textbf{输出}} % 替换关键词

    \Input{
        \\
        步长 $\eta$\;\\
        迭代数 $T$\;\\
        batch size $B$ \;\\
        初值 $\hat{x_0}$\;\\}
    \Output{
        \\
        $\hat{x_T}$\;\\}
    \BlankLine

    \For {$j=1,2,\cdots,T$}{
        Uniform sample a batch $I_j\in {1,2,\cdots,n}.$其中$|I_j|=B$\;
        $g_j=\nabla f(\hat{x_{j-1}})=\frac{1}{n}\sum_{i\in I_j}\nabla f_i(\hat{x_{j-1}})$ \;
        $x_0^{(j)}\leftarrow \hat{x_{j-1}}$ \;
        $N_j \leftarrow B $\;
        \For {$k=1,2,\cdots,N_j$}{
            Random pick $i_k\in I_j$\;
            $\nu_{k-1}^{(j)}\leftarrow \nabla f_{i_k}(x_{k-1}^{(j)})-f_{i_k}(x_{0}^{(j)})+g_j$\;
            $ x_k^{(j)\leftarrow x_{k-1}^{(j)}}-\eta \nu_{k-1}^{(j)} $ \;
        }
        $\hat{x_j}\leftarrow x_{N_j}^{(j)}$\;
    }
    \caption{Stochastically Controlled Stochastic Gradient (SCSG)\label{al3}}
\end{algorithm}

\subsection{计算细节}
在计算函数值以及梯度时会遇到如下两类问题
\begin{itemize}
  \item 计算$\frac{e^x}{1+e^x}$时$e^x=inf$
  \item 计算$log(1+e^x)$时$e^x=inf$
\end{itemize}
我们将其处理成$\frac{e^x}{1+e^x}=1,log(1+e^x)=x$
\section{计算结果}

对$\lambda=10,1,0.1,0.001,\frac{1}{n}$进行了实验
\paragraph{参数}
SVRG中$m=\frac{n}{10}$,SVRG以及SAG中$\alpha=0.01$,SVRG一共进行了20次迭代,SAG进行了n次迭代,sgBB进行了2n次迭代.
\paragraph{误差定义}
参考文献定义Testing Error$R(w)=\sum_{i=1}^{n}log(1+exp(-b^iw^Ta^i))$
\paragraph{结果取样}
由于R(w)的计算量太大,在展示结果时,我们只等距的选取了20个点的值来作图
\paragraph{x轴}
仿照参考文献,做了三类图,其x轴分别为迭代次数,求导次数,时间.在这里求导次数定义为单个$f_i$的求导次数.所以SVRG一次迭代求导次数了增加n+m次.
\paragraph{图注}
在每一组图的图注位置,在相应算法后记录最后时刻的R(w)

\subsection{MINIST}

\subsection{COVERTYPE}




\subsection{结果分析}


\bibliographystyle{plain}
\bibliography{1}
\end{document} 